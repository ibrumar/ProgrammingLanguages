\documentclass{article}
\usepackage[utf8]{inputenc}
\usepackage{plcourse}
\usepackage{ttquot}
\usepackage{proof}
\usepackage{stmaryrd}
\usepackage{angle}
\usepackage{amssymb}
\usepackage{hyperref}
\usepackage{graphicx}
\usepackage{tikz}
\usepackage{tikz-qtree}
\usepackage{xspace}
\usepackage{floatflt,amsmath,amssymb}
\usepackage[ligature, inference]{semantic}

\homework{1}

\newcommand{\largestep}[1]{\rulename{#1${}_{\text{Lrg}}$}}

\begin{document}
\hwsubheader


% %%%%%%%%%%%%%%%%%%%%%%%%%%
%
%    Cover page
%
%
\vspace{5cm}
{\LARGE
\begin{tabular}{rp{0.6\linewidth}}
  Name:&\todo{Iulian Valentin Brumar}\\
  HUID:&\todo{PUT YOUR HUID HERE}\\
  Collaborators:&{\normalsize
    \todo{Put your collaborators here, if any.}
    }
\end{tabular}
}

\vfill
\textit{Make sure that the remaining pages of this assignment do not contain any identifying information.}
\vfill


% %%%%%%%%%%%%%%%%%%%%%%%%%%
%
%    Question 1
%
%

\newpage
\begin{question}{Semantics}{(25 points)}
 
 \newcommand{\fact}[1]{\lamfnt{fact}~#1}
 \newcommand{\modulo}[2]{#1~\lamfnt{mod}~#2}

%%   Here are some useful macros/code for this questions. Try uncommenting them to see what they do.
  \begin{subquestion}
   In this exercise we have to extend the small step semantics to handle the factorial. \\

   We need a rule that takes an expression to a new one. The $\sigma$ might change since another expression might be an assignment that changes the store. \\
   
   \infrule[FactExp]
           {
             <e, \sigma> \stepsone <e', \sigma'>
           }
           {
             <\fact{e}, \sigma> \stepsone <\fact{e'}, \sigma'>
           }
           {
           }

    We do not include a rule that takes factorial(e) to factorial(n) since there is already a rule that maps expressions to naturals and we have defined above what happens in the general case of mapping an expression onto another expression. Next, we define an axiom for the factorial when it is applied to a number.


   \infrule[Fact]
           {
             %<e, \sigma> \stepsone <n, \sigma'>
           }
           {
             <\fact{n}, \sigma> \stepsone <m, \sigma>
           }
           {
             where m is the factorial of n
           }

   \end{subquestion}
   \begin{subquestion}
   Extend the small step semantics to handle modulo expressions. \\

   The rules are defined similarly to how we dealt with the factorial, but now with two operands and having to define the order of evaluation. We first evaluate the first operand of the modulo operation.

   \infrule[LMOD]
           {
             <e_1, \sigma> \stepsone <e_1', \sigma'>
           }
           {
             <\modulo{e_1}{e_2}, \sigma> \stepsone <\modulo{e_1'}{e_2}, \sigma'>
           }
           {
             %side condition
           }


   When the first operand is already an integer, we evaluate the second one.

   \infrule[RMOD]
           {
             <e_2, \sigma> \stepsone <e_2', \sigma'>
           }
           {
             <\modulo{n}{e_2}, \sigma> \stepsone <\modulo{n}{e_2'}, \sigma'>
           }
           {
             %where m is the factorial of n
           }


   When both operands are integers, the axiomatic rule is:

   \infrule[MOD]
           {
             %<e, \sigma> \stepsone <n, \sigma'>
           }
           {
             <\modulo{n}{m}, \sigma> \stepsone <k, \sigma>
           }
           {
             where k is n modulo m and $m > 0$.
           }
\end{subquestion}

\begin{subquestion}
What happens when we try to evaluate the expression $\modulo{(\fact{3})}{(-7)}$ having a store $\sigma_0$ that maps all the variables to 0.

The problem is that the second operand of the modulo operation is negative and the extension of the large step semantics defined in this exercise only allows for the second value to be strictly positive. The rule that we cannot apply is $MOD_{LRG}$.

Also our extension of the small step semantics results in the same constraint. The rule that we cannot apply is $MOD$.



%Let us first use the large step semantics:

%   \infrule[$MOD_{LRG}$]
%           {
             %<e, \sigma> \stepsone <n, \sigma'>

%             \infrule[$FACT_{LRG}$]
%                     {
%                        \infrule[$INT_{LRG}$]
%                        {}
%                        {<3, %\sigma_0> \stepsone <3, \sigma_0>}
                     
%                     }
%                     {
%                        %<\fact{3}, \sigma_0> \stepsto <6, \sigma_0>
%                     }  {}

%             \infrule[$INT_{LRG}$]
%                     {
                       %<3, %\sigma> %\stepsone %<3, %\sigma>
                     %}
                     %{
                     %  <-7, %\sigma_0> %\stepsto %<-7, %\sigma_0>
                     %}
                     %{}
          % }
          % {
          %   %<\modulo{(\fact{3})}{(-%7)}, \sigma_0> %\stepsto <1, \sigma_0>
          % }
          % {
          %   where 1 is 6 mod -7
          % }
    %Since there is no assignment in the derivation tree, sigma stays the same.

    %With the small step semantics the resulting derivations in order to prove that we can make the step $<\modulo{(\fact{3})}{(-7)}, \sigma_0> \stepsone <\modulo{(6)}{(-7)}, \sigma_0>$.


   %\infrule[LMOD]
   %        {
             %<e, \sigma> %\stepsone <n, %\sigma'>

%             \infrule[$Fact$]
%                     {
%                     }
%                     {
                        %<\fact{3}%, %\sigma_0>% %\stepsto %\sigma_0>
                    % }  {}
          % }
          % {
          %   %<\modulo{(\fact{3})}{(-7)}, \sigma_0> \stepsone<1, \sigma_0>
          % }
          % {
          %   %where 1 is 6 mod -7
          % }

%    Then to show that $<\modulo{(6)}{(-7)}, \sigma_0> \stepsone <1, \sigma_0>$.


%   \infrule[MOD]
%           {
             %<e, \sigma> \stepsone <n, \sigma'>
%           }
%           {
%             <\modulo{(\fact{3})}{(-7)}, %\sigma_0> \stepsone<1, \sigma_0>
 %          }
 %          {
 %            where 1 is 6 mod -7
 %          }

   %In both cases the $\sigma_0$ does not influence any of the derivations since there are no variable assignments involved.

\end{subquestion}
%   \infrule[FactExp]
%           {
%             a_1 \in A \quad a_2 \in A
%           }
%           {
%             a \in A
%           }
%           {
%             side condition
%           }
 

%  An example of inference rule
%  \infrule[M]{  \infrule[N]{A \quad B}{C}{}  \quad D}{E}{} 


%    \[
%    <e, \sigma> \stepsone <e', \sigma'>
%    \]
%
%    \[
%    <e, \sigma> \stepsto <n, \sigma'>
%    \]
%
% \[
% e::= \dots \mid \fact{e} \mid \modulo{e_1}{e_2}
% \]

          
          
  
\end{question}

% %%%%%%%%%%%%%%%%%%%%%%%%%%
%
%    Question 2
%
%

\newpage
\begin{question}{Binary Trees and Induction}{(25 points)}
\newcommand{\tree}[3]{(#1,#2,#3)}
\newcommand{\TreeSet}{\Set{Tree}\xspace}
\newcommand{\SixTree}{\Set{SixTree}\xspace}
\newcommand{\ROOT}{\ensuremath{\mathit{root}}}
\newcommand{\rootf}[1]{\ROOT(#1)}

  
%%   Here are some useful macros/code for this questions. Try uncommenting them to see what they do.
% $\rootf{\tree{4}{5}{6}}$



% $t_1=$ \Tree [.5 [.7 [.-1 ] [.3 ] ] [.4 ] ] &

% $t_1 \cong t_2$

% \begin{itemize}
% \item Case 1 where $e \equiv n$...
% \item Case 2 where $e %\equiv x$...
% \end{itemize}



  
  %% Question 2a
  \begin{subquestion}
    Write inference rules defining the relation $\cong$.

	We write a recursive and a base rule that consider two trees to have the same shape either if they both have two sub-trees or if they contain a single value. 

    \infrule[SameShapeRec]
	{
	%<e, \sigma> \stepsone <n, \sigma'>
		t1 \cong t1' t2 \cong t2'
	}
	{
	
	\tree{a}{t1}{t2} \cong \tree{a'}{t1'}{t2'} 
	%	$. (a, t_{1}, t_{2}) \cong \tree{a'}{t_1'}{t_2'}$ 
	}
	{
	where $a, a' \in \mathbb{Q}$
	}

    \infrule[SameShapeBase]
{
}
{
	
	a1 \cong a2
	%	$. (a, t_{1}, t_{2}) \cong \tree{a'}{t_1'}{t_2'}$ 
}
{
	where $a1, a2 \in \mathbb{Q}$
}
    
  \end{subquestion}

  %% Question 2b
  \begin{subquestion}
 Which of the following trees is are members of SixTree? For each such tree, provide a derivation
proving its membership in SixTree.


The tree A. is not a \textbf{SixTree} since $\tree{18}{6}{6}$ is not a base case (LEAF6 of LEAF18), it is not defined by the rule SumNode because $18 \ne 6 + 6$ and it is not defined by the rule ProdNode since $18 \ne 6\times6/6$.
  \end{subquestion}

Tree B is a SixTree. Next we write the proof:

%\infrule[SumNode]
%{
%	\infrule[ProdNode]
%	{
%		\infrule[LEAF6]
%		{
%		}
%		{
%			$6 \in SixTree$
%		}
%		{
%		}
%	
%		\infrule[LEAF6]
%		{
%		}
%		{
%		$6 \in SixTree$
%		}
%		{
%		}
%	}
%	{
%		$\tree{6}{6}{6}$
%	}
%	{
%	}
%
%	\infrule[LEAF18]
%	{
%	}
%	{
%	$18 \in SixTree$
%	}
%	{
%	}
%
%}
%{
%	$\tree{24}{\tree{6}{6}{6}}{18}$
%}
%{
%}

Tree C. is not even a complete tree since it contains a node with a 6 in its root but with only one leaf and trees are either defined as being one value or a value and two leaves (sub-trees).

Tree D. is not a SixTree because of the leaf with value 12. Since it is a single value, it cannot be produced by SumNode or by ProdNode. Also none of the axioms can generate it because 12 is not equal to 6 or 18.

  %% Question 2c
  \begin{subquestion}
 There are some trees in SixTree that have multiple derivations. That is, there are trees t such that there
are multiple and distinct derivations showing that $t \in SixTree$. Give one of these trees, and show two
distinct derivations of its membership in SixTree.

One of those SixTrees is 18. We know that because of the derivation LEAF18 and by considering the tree (18, (12, 6, 6), 6). The derivations for the latter are:

%\infrule[SumNode]
%{
%	\infrule[SumNode]
%	{
%		\infrule[LEAF6]
%		{
%		}
%		{
%			$6 \in SixTree$
%		}
%		{
%		}
%		
%		\infrule[LEAF6]
%		{
%		}
%		{
%			$6 \in SixTree$
%		}
%		{
%		}
%	}
%	{
%		$\tree{12}{6}{6}$
%	}
%	{
%	}
%	
%	\infrule[LEAF6]
%	{
%	}
%	{
%		$6 \in SixTree$
%	}
%	{
%	}
%	
%}
%{
%	$\tree{18}{\tree{12}{6}{6}}{6}$
%}
%{
%}
  \end{subquestion}

  %% Question 2d
  \begin{subquestion} 
    
    \begin{subsubquestion}
    What is the inductive reasoning principle for the set SixTree? Write it out in full.
         
    The inductive reasoning principle for the SixTree is;
    
    If: 
    
    \begin{itemize}
    	\item P(6) and P(18) are true.
    	\item $\forall t_1, t_2 \in SixTree$ if $P(t_1)$ and $P(t_2)$ is true then $P((a, t_1, t_2))$ is true, with $a = root(t_1) + root(t_2)$.
    	\item $\forall t_1, t_2 \in SixTree$ if $P(t_1)$ and $P(t_2)$ is true then $P((a, t_1, t_2))$ is true, with $a = \frac{root(t_1) \times root(t_2)}{6}$.
    \end{itemize}

	then:
	
	$\forall t \in SixTree P(t)$.

    \end{subsubquestion}
    \begin{subsubquestion}
What is the property that you are proving by induction? Write it out in full, in the form -For all
$t \in SixTree$, we will prove the property $P(t) = \dots$-.

We will prove $P(t) = \exists i \in \mathbb{Z}$ s.t. $root(t) = 6*i$. Note that this is a stronger statement that what is initially required by this question but if we can show that this is true, it will also be true that the root of any SixTree can only contain even numbers.
    
    \end{subsubquestion}
    \begin{subsubquestion}
 For each case mentioned in your inductive reasoning principle for SixTree, prove that case holds
for the property you stated above in Question 2(d).ii.
   
   For the rule LEAF6 we see that the property holds, i.e. $\exists i\in \mathbb{Z}$ s.t. $6 = i*6$. We pick $i = 1$.
   
   For the rule LEAF18 we see that the property holds as well, i.e. $\exists i\in \mathbb{Z}$ s.t. $18 = i*6$. We pick $i = 3$.
   
   This was the base case but let us now see the inductive case.
   
   We first focus on the inference rule SumNode. We first suppose that $\forall t_1, t_2 \in SixTree, \exists i,j \in \mathbb{Z}$  s.t. $root(t_1)=6*i$ and  $root(t_2)=6*j$. We have to show that $\exists k\in \mathbb{Z}$ s.t. $root((a, t1, t2)) = 6 \times k$. We have $root((a, t_1, t_2)) = a$ with $a = root(t_1) + root(t_2)$ according to the SumNode rule. $a = root(t_1) + root(t_2) = 6*i + 6*j = 6*(i + j)$. Therefore we can pick $k = i + j$.
   
   We next prove that the property works for ProdNode. We first suppose that $\forall t_1, t_2 \in SixTree, \exists i,j \in \mathbb{Z}$ s.t. $root(t_1)=6*i$ and  $root(t_2)=6*j$. We have to show that $\exists k \in \mathbb{Z}$ s.t. $root(()a, t1, t2)) = 6 \times k$. We have $root((a, t_1, t_2)) = a$ with $a = \frac{root(t_1) \times root(t_2)}{6}$ according to the ProdNode rule. $\frac{root(t_1) \times root(t_2)}{6} = \frac{6i \times 6j}{6} = 6ij$. Therefore we can pick $k = i*j$.
   
   
   Given that $\forall t \in SixTree. \exists \in \mathbb{Z}. root(t)=6k$ implies that $\forall t \in SixTree. \exists \in \mathbb{Z}. root(t)=2*(3*k)$. 
    \end{subsubquestion}
  \end{subquestion}

\end{question}

% %%%%%%%%%%%%%%%%%%%%%%%%%%
%
%    Question 3
%
%

\newpage
\begin{question}{Proof by Induction}{(25 points)}

%% Here are some useful macros/code for this questions. Try uncommenting them to see what they do.
%% $<e, \sigma> \stepsone^* <e',\sigma'> \stepsone^* <e'',\sigma''>$

  
  %% Question 3a
  \begin{subquestion}
  	 What is the inductive reasoning principle for the derivation of $<e, \sigma> \stepsone <e', \sigma'>$? Write it out in full.
  	 
  	 If
  	 \begin{itemize}
  	 \item $P(<x, \sigma> \stepsone <n, \sigma>)$ with $x \in Var$, $n \in \mathbb{Z}$ and $\sigma(x) = n$.
   	 \item $P(<i+k, \sigma> \stepsone <p, \sigma'>)$ with $i,k,p \in \mathbb{Z}$ and p being the sum of n and m.
   	 \item $P(<i \times k, \sigma> \stepsone <p, \sigma'>)$ with $i,k,p \in \mathbb{Z}$ and p being the product of n and m.
   	 \item $P(<x := k; e_2, \sigma> \stepsone <e_2, \sigma[x \rightarrow n]>)$.
  	 %\item $P(<e, \sigma> \stepsone <x, \sigma'>)$ with $x \in Var$.
%  	 \item $P(<e, \sigma> \stepsone <n, \sigma'>)$ with $n \in \mathbb{Z}$.
   	 \item $\forall e_1, e_2 \in AExp, \exists e_1' \in AExp$ if $P(<e_1, \sigma> \stepsone <e_1', \sigma'>)$ then $P(<e_1+e_2, \sigma> \stepsone <e_1'+e_2, \sigma'>)$.
   	 \item $\forall e_2 \in AExp$ and $, \forall k \in \mathbb{Z}$ if $P(<e_2, \sigma> \stepsone <e_2', \sigma'>)$ then $P(<k+e_2, \sigma> \stepsone <k+e_2', \sigma'>)$.
   	 \item $\forall e_1, e_2 \in AExp$ if $P(<e_1, \sigma> \stepsone <e_1', \sigma'>)$ then $P(<e_1 \times e_2, \sigma> \stepsone <e_1' \times e_2, \sigma'>)$.
   	 \item $\forall e_2 \in AExp$ and $, \forall k \in \mathbb{Z}$ if $P(<e_2, \sigma> \stepsone <e_2', \sigma'>)$ then $P(<k \times e_2, \sigma> \stepsone <k \times e_2', \sigma'>)$.
  	 %\item if $P(<e_1, \sigma> \stepsone <e_1', \sigma'>)$ $P(<x := e_1; e_2, \sigma> \stepsone <x := e_1'; e_2, \sigma'>)$.
  	 \end{itemize}
   	 then $\forall <e, \sigma> \in Config, \exists <e', \sigma'> \in Config, P(<e, \sigma> \stepsone <e', \sigma'>)$.
  	 
  	
  \end{subquestion}

  %% Question 3b
  \begin{subquestion}
    For all $<e,\sigma> \stepsone <e',\sigma'>$, we will prove the property $P(<e,\sigma> \stepsone <e',\sigma'>) =$ If $<e, \sigma> \stepsone <e',\sigma'> and <e', \sigma'> \stepsto <n, \sigma'>$ then $<e, \sigma> \stepsto <n, \sigma'>$.
  \end{subquestion}

  %% Question 3c
  \begin{subquestion}
  	
  	 For each case mentioned in your inductive reasoning principle for $<e, \sigma> \stepsone <e', \sigma'>$, prove that case
  	holds for the property you stated above. 


\begin{itemize}
%\item Case 1 where we have to show that $P(<e, \sigma> \stepsone <n, \sigma'>)$.

%The full statement we need to prove is: if $<e, \sigma> \stepsone <n, \sigma''>)$ and $<n, \sigma''> \stepsto <n, \sigma'>)$ then $<e, \sigma> \stepsto <n, \sigma'>$.

%We can see that $\sigma = \sigma''$ since the only rules with the conclusion $<e, \sigma> \stepsone <n, \sigma''>$ are VAR, ADD and MUL. In all those cases the store does not change.

%From $<n, \sigma''> \stepsto <n, \sigma'>)$ we find that $\sigma'' = \sigma'$. This is true since the only large step semantics rule with $<n, \sigma''> \stepsto <n, \sigma'>$ as a conclusion is $INT_{LRG}$ and in that rule the store does not change.

%Since $\sigma'' = \sigma'$, we can transform the hypothesis $<e, \sigma> \stepsone <n, \sigma''>$ to $<e, \sigma> \stepsone <n, \sigma'>$. This implies that $<e, \sigma> \stepsone^* <n, \sigma'>$ and given that $<n, \sigma'>$ is a final configuration we conclude that $<e, \sigma> \stepsto <n, \sigma'>)$.



\item Case 1 where we have to show that $P(<x, \sigma> \stepsone <n, \sigma'>)$.

The full statement we need to prove is: if $<x, \sigma> \stepsone <n, \sigma''>)$ and $<n, \sigma''> \stepsto <n, \sigma>)$ then $<x, \sigma> \stepsto <n, \sigma>$.

%We can see that $\sigma = \sigma''$ since the only rules with the conclusion $<e, \sigma> \stepsone <n, \sigma''>$ are VAR, ADD and MUL. In all those cases the store does not change.

From $<n, \sigma''> \stepsto <n, \sigma>)$ we deduce that $\sigma'' = \sigma$. This is true since the only large step semantics rule with $<n, \sigma''> \stepsto <n, \sigma>$ as a conclusion is $INT_{LRG}$.

Since $\sigma'' = \sigma$, we can transform the hypothesis $<x, \sigma> \stepsone <n, \sigma''>$ to $<x, \sigma> \stepsone <n, \sigma>$. The only inference rule in the small step semantics with this conclusion is VAR. This implies that $\sigma' = \sigma$ and that $n = \sigma(x)$. 

The conclusion that we try to prove which is $<x, \sigma> \stepsto <n, \sigma>$, can only be inferred by the large step semantics rule  $VAR_{LRG}$ and it requires that $\sigma = \sigma'$ which is the case as we have seen in the previous two paragraphs, and it requires that $\sigma(x) = n$ which is true as shown in the previous paragraph.

%From the previous paragraph we also know that $\sigma'(x) = n$, meaning that $<x, \sigma'> \stepsto <n, \sigma'>$ according to the small step semantics axiom VAR.


%Putting together the two small step semantics rules $<x, \sigma> \stepsone <n, \sigma'>$ and $<n, \sigma'> \stepsto <n, \sigma'>$ implies that $<x, \sigma> \stepsone^* <n, \sigma'>)$ and given that $<n, \sigma'>$ is a final configuration we conclude that $<x, \sigma> \stepsto <n, \sigma'>)$.

\item Case 2 The full statement we need to prove is: if $<i+k, \sigma> \stepsone <p, \sigma''>)$ and $<p, \sigma''> \stepsto <n, \sigma>)$ then $<i+k, \sigma> \stepsto <n, \sigma>$.

From $<p, \sigma''> \stepsto <n, \sigma>)$ we deduce that $\sigma'' = \sigma$ and that $n = p$. This is true since the only large step semantics rule with $<p, \sigma''> \stepsto <n, \sigma>$ as a conclusion is $INT_{LRG}$ and it has the above mentioned constraints.

What we try to prove is $<i+k, \sigma> \stepsto <n, \sigma>$ and we have seen in the previous paragraph that $n = p$ and we also know that $p = i + k$ because $<i+k, \sigma> \stepsone <p, \sigma''>)$. We can therefore transform the statement we want to prove to $<n, \sigma> \stepsto <n, \sigma>$ which is true according to large step semantics axiom $VAR_{LRG}$.

\item Case 3 The full statement we need to prove is: if $<i \times k, \sigma> \stepsone <p, \sigma''>)$ and $<p, \sigma''> \stepsto <n, \sigma>)$ then $<i \times k, \sigma> \stepsto <n, \sigma>$.

As in case 2, from $<p, \sigma''> \stepsto <n, \sigma>)$ we deduce that $\sigma'' = \sigma$ and that $n = p$. 

What we try to prove is $<i \times k, \sigma> \stepsto <n, \sigma>$ and we have seen in the previous paragraph that $n = p$ and we also know that $p = i \times k$ because $<i+k, \sigma> \stepsone <p, \sigma''>)$. We can therefore transform the statement we want to prove to $<n, \sigma> \stepsto <n, \sigma>$ which is true according to large step semantics axiom $VAR_{LRG}$.


\item Case 4. The full statement we need to prove is: if $<x := k; e_2, \sigma> \stepsone <e_2, \sigma''>)$ and $<e_2, \sigma''> \stepsto <n, \sigma[x \rightarrow k]>)$ then $<x := k; e_2, \sigma''> \stepsto <n, \sigma[x \rightarrow k]>$. 

The only rule in the large step semantics that considers assignment in its conclusion is $ASG_{LRG}$ and it has the following form:

%uncomment this when sending
%\infrule[$INT_{LRG}$]
%{
%	<e_1, \sigma_2> \stepsto <n_1, \sigma_2''> <e_2, \sigma_2'' $[x \rightarrow	n_1]$> \stepsto <n_2, \sigma_2'>
%}
%{
%	<x := e_1;e_2, \sigma_2> \stepsto <n_2, \sigma_2'>
%}
%{}

In this context, $e_1$ is mapped to k, n1 is mapped to k as well, n2 is mapped to n, $\sigma_2$ and $\sigma_2''$ to $\sigma$ and $\sigma_2'$ to $\sigma[x \rightarrow k]$. If we rewrite the rule in this context:

%uncomment this when sending
%\infrule[$INT_{LRG}$]
%{
	%$<k, \sigma> \stepsto <k, %\sigma> <e_2, \sigma[x %\rightarrow k]> \stepsto <n, %\sigma[x \rightarrow k]>$
%}
%{
%	<x := k;e_2, \sigma> \stepsto <n, \sigma[x \rightarrow k]>
%}
%{
%hola
%}

The premise $<k, \sigma> \stepsto <k, \sigma>$ is true by the axiom $INT_{LRG}$ in the large step semantics. 

We only have to guarantee that $<e_2, \sigma[x \rightarrow k]> \stepsto <n, \sigma[x \rightarrow n]>$. We will deduce this from $<x := k; e_2, \sigma> \stepsone <e_2, \sigma''>)$ and $<e_2, \sigma''> \stepsto <n, \sigma[x \rightarrow k]>)$.

From $<x := k; e_2, \sigma> \stepsone <e_2, \sigma''>$ we deduce that $\sigma'' = \sigma[x \rightarrow k]$ according to rule ASG in the small step semantics. For this reason we can rewrite $<e_2, \sigma''> \stepsto <n, \sigma[x \rightarrow k]>)$ as $<e_2, \sigma[x \rightarrow k]> \stepsto <n, \sigma[x \rightarrow k]>)$ which is exactly the second premise of the $INT_{LRG}$ rule.



%still have to complete everything else after case 3.
\item Case 5. We want to show that $\forall e_1 \in AExp$ $P(<e_1, \sigma> \stepsone <e_1', \sigma'>)$ then $P(<e_1+e_2, \sigma> \stepsone <e_1'+e_2, \sigma'>)$. More precisely we want to show that if $<e_1+e_2, \sigma> \stepsone <e_1'+e_2, \sigma''>$ and $<e_1'+e_2, \sigma''> \stepsto <n, \sigma'>$ (2) then $<e_1+e_2, \sigma> \stepsto <n, \sigma'>$.

If we knew that $e_1'$ in $<e_1'+e_2, \sigma''> \stepsto <n, \sigma'>$, we could map $e_2$ to the second premise in the large step semantics rule $ADD_{LRG}$ with $n_2 = n - n_1$ (we will later get $n_1$ through the inductive hypothesis and n is what $e_1' + e_2$ is mapped to by the large step semantics hypothesis (2)):

\infrule[$ADD_{LRG}$]
{
	<e_1, \sigma> \stepsto <n_1, \sigma''> <e_2, \sigma''> \stepsto <n_2, \sigma'>
}
{
	<e_1 + e_2, \sigma> \stepsto <n, \sigma'>
}
{
	where n is the sum of k and $n_2$
}

We get the first premise through the inductive hypothesis. We know that $P(<e_1, \sigma_a>  \stepsone <e_1', \sigma_a'>)$ which means: $<e_1, \sigma_a> \stepsone <e_1', \sigma_a''>$ and $<e_1', \sigma_a''> \stepsto <k, \sigma_a'>$ then $<e_1, \sigma_a> \stepsto <k, \sigma_a'>$. We can then map k to $n_1$ in the $ADD_{LRG}$ rule above, and $\sigma''$ to $\sigma_a'$. We take $\sigma = \sigma_a$ and $n_2 = n - k$.

We can now rewrite the $ADD_{LRG}$ rule like this:

\infrule[$ADD_{LRG}$]
{
	<e_1, \sigma_a> \stepsto <k, \sigma_a'> <e_2, \sigma_a'> \stepsto <n_2, \sigma'>
}
{
	<e_1 + e_2, \sigma_a> \stepsto <n, \sigma'>
}
{
	Where n is the sum of k and $n_2$
}

\item Case 6. We want to show that $\forall e_2 \in AExp, k \in \mathbb{Z}$ $P(<e_2, \sigma> \stepsone <e_2', \sigma'>)$ then $P(<k+e_2, \sigma> \stepsone <k'+e_2', \sigma'>)$. More precisely we want to show that if $<k+e_2, \sigma> \stepsone <k+e_2', \sigma''>$ (1) and $<k+e_2', \sigma''> \stepsto <n, \sigma'>$ (2) then $<k+e_2, \sigma> \stepsto <n, \sigma'>$.


We can get to the conclusion we want by using the large step semantics rule $ADD_{LRG}$:

\infrule[$ADD_{LRG}$]
{
	<k, \sigma> \stepsto <k, \sigma> <e_2, \sigma> \stepsto <n_2, \sigma'>
}
{
	<k + e_2, \sigma> \stepsto <n, \sigma'>
}
{
	with n being the sum of k and $n_2$
}

The first premise comes for free thanks to the axiom $INT_{LRG}$ and the second premise comes through the inductive hypothesis. By IH, we know that $P(<e_2, \sigma_a>  \stepsone <e_2', \sigma_a'>)$ which means: $<e_2, \sigma_a> \stepsone <e_2', \sigma_a''>$ and $<e_2', \sigma_a''> \stepsto <h, \sigma_a'>$ then $<e_2, \sigma_a> \stepsto <h, \sigma_a'>$. We can then map h to $n_2$ in the $ADD_{LRG}$ rule above. We take $\sigma = \sigma_a$.

We can now rewrite the $ADD_{LRG}$ rule like this:

\infrule[$ADD_{LRG}$]
{
	<k, \sigma> \stepsto <k, \sigma> <e_2, \sigma> \stepsto <h, \sigma'>
}
{
	<k + e_2, \sigma> \stepsto <n, \sigma'>
}
{
	with n being the sum of k and h
}



\item Case 7 We want to show that $\forall e_1 \in AExp$ $P(<e_1, \sigma> \stepsone <e_1', \sigma'>)$ then $P(<e_1 \times e_2, \sigma> \stepsone <e_1' \times e_2, \sigma'>)$. More precisely we want to show that if $<e_1 \times e_2, \sigma> \stepsone <e_1' \times e_2, \sigma''>$ and $<e_1' \times e_2, \sigma''> \stepsto <n, \sigma'>$ (2) then $<e_1 \times e_2, \sigma> \stepsto <n, \sigma'>$.

If we knew that $e_1'$ in $<e_1'  \times e_2, \sigma''> \stepsto <n, \sigma'>$, we could map $e_2$ to the second premise in the large step semantics rule $MUL_{LRG}$ with $n_2 = \frac{n} {n_1}$ (we will later get $n_1$ through the inductive hypothesis and n is what $e_1' \times e_2$ is mapped to by the large step semantics hypothesis (2)):

\infrule[$MUL_{LRG}$]
{
	<e_1, \sigma> \stepsto <n_1, \sigma''> <e_2, \sigma''> \stepsto <n_2, \sigma'>
}
{
	<e_1 \times e_2, \sigma> \stepsto <n, \sigma'>
}
{
	where n is the product of k and $n_2$
}

We get the first premise through the inductive hypothesis. We know that $P(<e_1, \sigma_a>  \stepsone <e_1', \sigma_a'>)$ which means: $<e_1, \sigma_a> \stepsone <e_1', \sigma_a''>$ and $<e_1', \sigma_a''> \stepsto <k, \sigma_a'>$ then $<e_1, \sigma_a> \stepsto <k, \sigma_a'>$. We can then map k to $n_1$ in the $MUL_{LRG}$ rule above, and $\sigma''$ to $\sigma_a'$. We take $\sigma = \sigma_a$ and $n_2 = n / k$.

We can now rewrite the $MUL_{LRG}$ rule like this:

\infrule[$MUL_{LRG}$]
{
	<e_1, \sigma_a> \stepsto <k, \sigma_a'> <e_2, \sigma_a'> \stepsto <n_2, \sigma'>
}
{
	<e_1 \times e_2, \sigma_a> \stepsto <n, \sigma'>
}
{
	Where n is the product of k and $n_2$
}

\item Case 8 We want to show that $\forall e_2 \in AExp, k \in \mathbb{Z}$ $P(<e_2, \sigma> \stepsone <e_2', \sigma'>)$ then $P(<k \times e_2, \sigma> \stepsone <k' \times e_2', \sigma'>)$. More precisely we want to show that if $<k \times e_2, \sigma> \stepsone <k \times e_2', \sigma''>$ (1) and $<k \times e_2', \sigma''> \stepsto <n, \sigma'>$ (2) then $<k \times e_2, \sigma> \stepsto <n, \sigma'>$.


We can get to the conclusion we want by using the large step semantics rule $MUL_{LRG}$:

\infrule[$MUL_{LRG}$]
{
	<k, \sigma> \stepsto <k, \sigma> <e_2, \sigma> \stepsto <n_2, \sigma'>
}
{
	<k \times e_2, \sigma> \stepsto <n, \sigma'>
}
{
	with n being the sum of k and $n_2$
}

The first premise comes for free thanks to the axiom $INT_{LRG}$ and the second premise comes through the inductive hypothesis. By IH, we know that $P(<e_2, \sigma_a>  \stepsone <e_2', \sigma_a'>)$ which means: $<e_2, \sigma_a> \stepsone <e_2', \sigma_a''>$ and $<e_2', \sigma_a''> \stepsto <h, \sigma_a'>$ then $<e_2, \sigma_a> \stepsto <h, \sigma_a'>$. We can then map h to $n_2$ in the $ADD_{LRG}$ rule above. We take $\sigma = \sigma_a$.

We can now rewrite the $MUL_{LRG}$ rule like this:

\infrule[$MUL_{LRG}$]
{
	<k, \sigma> \stepsto <k, \sigma> <e_2, \sigma> \stepsto <h, \sigma'>
}
{
	<k + e_2, \sigma> \stepsto <n, \sigma'>
}
{
	with n being the sum of k and h
}

%\item Case 9. The full statement we need to prove is: if $<x := e_1; e_2, \sigma> \stepsone <x := e_1'; e_2, \sigma''>$ and $<x := e_1'; e_2, \sigma''> \stepsto <n, \sigma'>)$ then $<x := e_1; e_2, \sigma''> \stepsto <n, \sigma'>$. 

%The only rule in the large step semantics that considers assignment in its conclusion is $ASG_{LRG}$ and it has the following form:

%uncomment this when sending
%\infrule[$INT_{LRG}$]
%{
%	<e_1, \sigma_2> \stepsto <n_1, \sigma_2''> <e_2, \sigma_2'' $[x \rightarrow	n_1]$> \stepsto <n_2, \sigma_2'>
%}
%{
%	<x := e_1;e_2, \sigma_2> \stepsto <n_2, \sigma_2'>
%}
%{
%}

%In this context, $e_1$ is mapped to k, n1 is mapped to k as well, n2 is mapped to n, $\sigma_2$ and $\sigma_2''$ to $\sigma$ and $\sigma_2'$ to $\sigma[x \rightarrow k]$. If we rewrite the rule in this context:

%uncomment this when sending
%\infrule[$INT_{LRG}$]
%{
%$<k, \sigma> \stepsto <k, %\sigma> <e_2, \sigma[x %\rightarrow k]> \stepsto <n, %\sigma[x \rightarrow k]>$
%}
%{
%	<x := k;e_2, \sigma> \stepsto <n, \sigma[x \rightarrow k]>
%}
%{
%hola
%}

%The premise $<k, \sigma> \stepsto <k, \sigma>$ is true by the axiom $INT_{LRG}$ in the large step semantics. 

%We only have to guarantee that $<e_2, \sigma[x \rightarrow k]> \stepsto <n, \sigma[x \rightarrow n]>$. We will deduce this from $<x := k; e_2, \sigma> \stepsone <e_2, \sigma''>)$ and $<e_2, \sigma''> \stepsto <n, \sigma[x \rightarrow k]>)$.

%From $<x := k; e_2, \sigma> \stepsone <e_2, \sigma''>$ we deduce that $\sigma'' = \sigma[x \rightarrow k]$ according to rule ASG in the small step semantics. For this reason we can rewrite $<e_2, \sigma''> \stepsto <n, \sigma[x \rightarrow k]>)$ as $<e_2, \sigma[x \rightarrow k]> \stepsto <n, \sigma[x \rightarrow k]>)$ which is exactly the second premise of the $INT_{LRG}$ rule.



\end{itemize}
    
  \end{subquestion}

\end{question}





% %%%%%%%%%%%%%%%%%%%%%%%%%%
%
%    Question 4
%
%

\newpage
\begin{question}{Stack-Oriented Programming}{(25 points)}

\newcommand{\push}[1]{\ensuremath{\mathbf{push}\;#1}}
\newcommand{\add}{\ensuremath{\mathbf{add}}}
\newcommand{\mul}{\ensuremath{\mathbf{mul}}}
\newcommand{\divcom}{\ensuremath{\mathbf{div}}}
\newcommand{\pop}{\ensuremath{\mathbf{pop}}}
\newcommand{\dup}{\ensuremath{\mathbf{dup}}}
\newcommand{\swap}[1]{\ensuremath{\mathbf{swap}\;{#1}}}
\newcommand{\seq}[2]{\ensuremath{#1; #2}}
\renewcommand{\skip}{\ensuremath{\mathbf{skip}}}
\newcommand{\WHILENE}{\ensuremath{\mathbf{whilene}}\xspace}
\newcommand{\whilene}[1]{\ensuremath{\WHILENE\;\{#1\}}}
\newcommand{\emptystack}{\ensuremath{[]}}
\newcommand{\stack}{\ensuremath{S}}
\newcommand{\cons}[2]{\ensuremath{#1 :: #2}}

%% Here are some useful macros/code for this questions. Try uncommenting them to see what they do.
% $<\push{3}; \pop; \whilene{\push{6};\push{7};\mul}, \stack>$

% $<\push{8},S> \stepsto \cons{8}{S}$


  \begin{subquestion}
    \begin{subsubquestion}
      Command $\push{n}$: 
      
      \infrule[PUSH]
      {
      	
      }
      {
      	<$\push n$, $\stack$> \stepsto $\cons{n}{S}$
      }
      {
      	%where m is the factorial of n
      }
    \end{subsubquestion}
    \begin{subsubquestion}
      Command $\pop$: No more inference rules to add.
    \end{subsubquestion}
    \begin{subsubquestion}
      Command $\add$:
      
      \infrule[ADD]
      {
      	
      }
      {
      	<$\add$, $\cons{\cons{n}{m}}{S}$> \stepsto $\cons{k}{S}$
      }
      {
      	where k is the sum of n and m.
      }
    \end{subsubquestion}
    \begin{subsubquestion}
      Command $\mul$: 
      
      \infrule[MUL]
      {
      	
      }
      {
      	<$\mul$, $\cons{\cons{n}{m}}{S}$> \stepsto $\cons{k}{S}$
      }
      {
      	where k is the multiplication of n and m.
      }
    \end{subsubquestion}
    \begin{subsubquestion}
      Command $\divcom$: No more inference rules to add.
    \end{subsubquestion}
    \begin{subsubquestion}
      Command $\dup$: 
      
      \infrule[DUP]
      {
      	
      }
      {
      	<$\dup$, $\cons{n}{S}$> \stepsto $\cons{n}{\cons{n}{S}}$
      }
      {
      	%where k is the multiplication of n and m.
      }
    \end{subsubquestion}
    \begin{subsubquestion}
      Command $\swap{i}$: No more inference rules to add.
    \end{subsubquestion}
    \begin{subsubquestion}
      Command $\seq{c_1}{c_2}$: No more inference rules to add.
    \end{subsubquestion}
    \begin{subsubquestion}
      Command $\whilene{c}$: 
      
      \infrule[WHILENE2]
      {
      	<$c, \cons{\cons{n_0}{n_1}}{S}$> \stepsto $\stack''$ <$\whilene{c}$, $\stack''$> \stepsto $\stack'$
      }
      {
      	<$\whilene{c}$, $\cons{n_0}{\cons{n_1}{S}}$> \stepsto $\stack'$
      }
      {
      	$n_0 \ne n_1$
      }
      
    \end{subsubquestion}
    \begin{subsubquestion}
      Command $\skip$: 
      
      \infrule[SKIP]
      {
      	
      }
      {
      	<$\skip$, $\stack$> \stepsto $\stack$
      }
      {
      	%where k is the multiplication of n and m.
      }
    \end{subsubquestion}

  \end{subquestion}

  % Question 4b
  \begin{subquestion}
     Give a configuration that gets stuck. That is, give a configuration <c, \stack>, such that there does not exist a stack S' where $<c, S> \stepsto S'$.
     
     An infinite loop that keeps modifying the stack will prevent the election of a fixed target S'. An example would be $\whilene{\push{1}}, []$.
    
  \end{subquestion}
  
  % Question 4c
  \begin{subquestion}
 Write a program in our toy language that takes as input a stack of the form n :: [] where $n \geq 0$ and
finishes with a stack of the form 0 :: 1 :: 2 :: \dots :: n :: []. Briefly describe (in words) how your
implementation works. (That is, help the grader understand why your code is correct!)

The program we propose is:\\
 \emph{push 0; whilene \{ pop; dup; push -1; add; push 0 \}; pop}
 
 The idea is to keep a 0 on top of the stack before checking the condition of the while command, and to leave the number i+1 just underneath the top element (0) so that we can remember at each iteration to leave the element i (also under the top which should be 0).
 
 The reason for leaving a 0 on top of the stack is because the algorithm must finish when the i'th element we have to compute is 0. The reason why we do a dup at the beginning of each iteration just before the pop of the top 0 value, is because we do not want to lose the i+1 value when computing the sum: we just want to subtract 1 from it for the current iteration.
 
 Obviously before finishing the program there is a redundant zero on top of the stack which we eliminate with a pop. 


  \end{subquestion}
  
  % Question 4d
  \begin{subquestion}
    Write a program in our toy language that takes as input a stack of the form n :: [] where $n \geq 0$ and
   finishes with a stack of the form n :: (n - 1) :: \dots :: 1 :: 0 :: []. Briefly describe (in words) how your
   implementation works. (That is, help the grader understand why your code is correct!)
   
   The implementation we propose is:
   
   \emph{push 0; whilene \{ dup; push 1; add; swap 1; swap 2; swap 1 \}; pop}.
   
   We first push on top of the stack a 0 and in general, before checking whilene's condition, we guarantee that on top of the stack there is the value i-1 computed during the previous iteration and the second element is n so that the loop stops when i-1 is n.
   
   Guaranteeing this ordering of the values i-1 as the top element and of n as the second requires the manipulation with the three swaps we have in whilene's loop body. The first three instructions in the loop body leave us with i, i-1 and n on top of the stack. The first swap changes the order to i-1, i and n; the second swap to n, i, i-1 and the last swap to i, n, i-1. If n turns out to be equal to n, the loop will stop and we have to pop a redundant n value on top.
    
  \end{subquestion}
  
  
  
  % Question 4e
  \begin{subquestion}
     Write a program that takes as input a stack of the form m :: n :: [] where $m \gt 0$ and $n \gt 0$ and finishes
    with a stack of the form i :: [], where i is 0 if m is a multiple of n, and 1 otherwise. Briefly describe
    (in words) how your implementation works. (That is, help the grader understand why your code is
    correct!).
    
    The solution we propose is:
    
    \emph{dup; swap 2; dup; swap 2; div; mul; div; ; push -1; mul; push 1; add}.
    
    The program is divided in two parts. The last div outputs a 1 if m is a multiple of n and a 0 if it is not. The next 4 instructions just invert 0 for a 1 and 1 for a 0.
    
    In the first four instructions before the first division leave the stack in a format of m :: n :: n :: m. The reason for this is that we first want to divide m by n to produce q. We know that if we multiply q by n the result is smaller or equal to m. If it is not equal, it means that m is not divisible by n. The next step is to divide n*q by m. If the result is one it means that they were equal but the integer division will return zero otherwise.
    
    In the second part in order to invert the result 'p' in the previous paragraph , we change its sign (by multiplying with -1) and add one to that. If p was 0, changing the sign has no effect and by adding one, the overall result is one. If p was 1, changing the sign results in -1 and adding 1 to that gives 0.
    
    
  \end{subquestion}
  
    
\end{question}



\end{document}
